\begin{abstract}
It has become exceedingly simple to communicate and share information thanks to inventions such like personal computers and the Internet. Moreover it has become trivial to just copy the raw information and leave out annoyances such as references as to where the information originated from or who is the author, a practice which in one word can be summarized as plagiarism. This practice is dreaded in academia and the industry for obvious reasons making the idea of automatically detecting it very attractive. 

This article presents a hierarchy of the types of plagiarism that can arise in documents, as well as known algorithmic methods for detecting them. The main focus lies on detecting signs indicating possible plagiarism in scientific articles. On a sideline we will cover some methods for detecting plagiarism in the source code of computer programs. Finally we'll look into a few of the available software packages that promise to simplify the process of review in academia.
\end{abstract}
